% chapter 5
\chapter{总结与展望}
\section{论文工作总结}
在本文中系统研究了基于咳嗽音频的哮喘检测方式,由于系统的检测方式是基于Android平台搭建的,而不需要专业的检测机器,因此对比其他很多的哮喘诊断方法具有更好的实用性和普查性。系统再此研究的背景下研究了音频中咳嗽事件的检测、迁移学习实现特征提取以及支持向量机份分类的性能问题。

对于音频中咳嗽事件的检测问题,系统提出建立一个CNN网络进行咳嗽事件的检测。该网络是利用Kvapilova等人数据库的采集咳嗽样本加以环境噪音进行训练的。系统通过分析网络参数的重要性和相关性后,将网络性能的重点聚焦于特征子图的个数以及损失函数的选取。最终通过分析确定了损失函数为交叉熵损失函数,特征子图个数为16,16,32,32,并经过联合分类决策,极大地提高了检测网络的可靠性。

对于迁移学习问题,系统使用Youtube数据集训练的类神经网络VGGish提取咳嗽音频数据的特征向量,但是因为VGGish仅基于频谱图输入进行特征提取,会遗漏时域中的一些重要特征,因此系统还采用了人工设计的方式提取了新的500维特征向量,保证了特征空间的全面性和准确性

最后对于支持向量机的分类性能问题,系统主要分析了线性支持向量机和带RBF核的支持向量机(RBF-SVM)的性能比较关系。由于系统提取到的特征向量维度较高且重点不突出,系统采取PCA方法对特征向量进行降维处理,并分析不同训练比对两类向量机性能的影响,最终系统选取了训练集-测试集80:20的情况并采用RBF-SVM进行特征分类,达到了90\%以上的分类效率。

综上所述本研究主要实现一种基于咳嗽音频的哮喘检测方式,通过手机麦克风收集用户的咳嗽音频,并在经过音频数据增强后,利用联合特征提取和VGG模型特征提取,提取一个733维度的特征向量,将所有的特征点进行主成分分析法(PCA)降维处理后,使RBF支持向量机进行二分分类,分类准确率平均达到90\%以上。

\section{下一步工作}
虽然本论文的最终分类准确率达到了90\%以上,但是本研究的局限性也很大。首先是音频提取的负样本不够全面,没有办法很好的涵盖各种环境下咳嗽音的检测,在环境噪音较大的情况下决策失败率很高,需要多次采样。未来改进的方法主要有以下两种:1.使用VGGish模型对咳嗽事件检测采取小样本学习;2.继续加大负样本量,通过数据增强等方式,提高模型的鲁棒性

其次在哮喘检测的工作中,只是简单的对哮喘健康进行分类,而没有对其他异常咳嗽情况进行检测和分类。在测试样本中,有一名志愿者没有患有哮喘但是近期有咽炎症状,分类效果并不良好,因此在上面的分类实验中,已经去除该志愿者的实验数据。为了提高系统的冗余性,该研究的下一步工作就是去寻找其他几种常见异常咳嗽病人进行细分类。

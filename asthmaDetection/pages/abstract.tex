% 中文摘要
智能手机和可穿戴设备的人体健康检测系统已成为诊断医学和计算机应用交汇的热点。文研究主要利用音频建模技术对哮喘这一种社会常见呼吸道疾病进行诊断。文本对国内外人体信号检测系统进行充分的研究,发现移动设备健康检测作为一种新兴的监测手段,具有易测量、诊断时间短、特异性强等显著优势,有很强的实用价值。

本研究基于以上背景,实现了一种基于咳嗽音频分析哮喘检测系统。该系统利用移动设备的麦克风以44.1Khz的采样率收集用户的咳嗽音频,并通过咳嗽检测网络检测音频中的咳嗽事件并选择样本质量较好的咳嗽音频。接着,本系统将过滤后的咳嗽音频重采样到22Khz,通过音频增强技术扩增咳嗽音频的数据量,并提高整体系统对复杂情况的鲁棒性。之后系统按照人工设计特征提取和VGG预训练模型迁移学习特征提取两种方式,对每个单周期咳嗽音频提取733维度的特征向量。在对提取的特征向量进行主成分分析法完成降维处理后,系统对所有的特征向量采取RBF-SVM进行二分分类,并验证分类效果。

在实际的检测环节,由于咳嗽检测网络通过单周期多层决策提高决策正确率,在环境噪音大的情况下,决策失败较高,往往需要多次样本收集才能完成咳嗽检测的决策。

实验阶段实验组采集了4名患有哮喘的病人采集了10min左右的咳嗽信号筛选了共300组单周期咳嗽信号,同时采集了健康人的500组单周期咳嗽信号。利用这些数据研究了音频中咳嗽事件的检测、利用迁移学习实现特征提取以及支持向量机份分类的性能问题。最终优化系统网络参数,选取了训练集-测试集比率为80:20的情况并采用 RBF-SVM 进行特征分类,达到了 91.61\% 的分类正确率。


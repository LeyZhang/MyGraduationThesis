% 英文摘要
The human health detection system of smart phones and wearable devices has become a hot spot at the intersection of diagnostic medicine and computer applications. This research mainly uses audio modeling technology to diagnose asthma, a common respiratory disease in society. The text conducted a thorough research on human body signal detection systems at home and abroad, and found that mobile device health detection, as an emerging monitoring method, has significant advantages such as easy measurement, short diagnosis time, and strong specificity, and has strong practical value.

Based on the above background, this research has implemented an asthma detection system based on cough audio analysis. The system uses the microphone of the mobile device to collect the user's cough audio at a sampling rate of 44.1Khz, and detects cough events in the audio through the cough detection network and selects cough audio with better sample quality. Next, the system resamples the filtered cough audio to 22Khz, amplifies the data volume of the cough audio through audio enhancement technology, and improves the robustness of the overall system to complex situations. After that, the system extracts a 733-dimensional feature vector for each single-cycle cough audio according to two methods: manual design feature extraction and VGG pre-training model migration learning feature extraction. After performing the principal component analysis method on the extracted feature vectors to complete the dimensionality reduction processing, the system adopts RBF-SVM for binary classification of all feature vectors, and verifies the classification effect.

In the actual detection link, because the cough detection network improves the accuracy of decision-making through single-cycle multi-layer decision-making, decision-making failures are high in the case of large environmental noise, and it often requires multiple sample collections to complete the cough detection decision-making.

In the experimental stage, the experimental group collected 4 patients with asthma, collected cough signals for about 10 minutes, screened a total of 300 groups of single-cycle cough signals, and collected 500 groups of single-cycle cough signals from healthy people. Using these data, the performance problems of detecting cough events in audio, using transfer learning to achieve feature extraction, and support vector machine classification are studied. Finally, the system network parameters were optimized, the training set-test set ratio was selected as 80:20, and RBF-SVM was used for feature classification, achieving a classification accuracy rate of 91.61\%.